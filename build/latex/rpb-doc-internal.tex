%% Generated by Sphinx.
\def\sphinxdocclass{report}
\documentclass[letterpaper,10pt,english]{sphinxmanual}
\ifdefined\pdfpxdimen
   \let\sphinxpxdimen\pdfpxdimen\else\newdimen\sphinxpxdimen
\fi \sphinxpxdimen=.75bp\relax

\PassOptionsToPackage{warn}{textcomp}
\usepackage[utf8]{inputenc}
\ifdefined\DeclareUnicodeCharacter
 \ifdefined\DeclareUnicodeCharacterAsOptional
  \DeclareUnicodeCharacter{"00A0}{\nobreakspace}
  \DeclareUnicodeCharacter{"2500}{\sphinxunichar{2500}}
  \DeclareUnicodeCharacter{"2502}{\sphinxunichar{2502}}
  \DeclareUnicodeCharacter{"2514}{\sphinxunichar{2514}}
  \DeclareUnicodeCharacter{"251C}{\sphinxunichar{251C}}
  \DeclareUnicodeCharacter{"2572}{\textbackslash}
 \else
  \DeclareUnicodeCharacter{00A0}{\nobreakspace}
  \DeclareUnicodeCharacter{2500}{\sphinxunichar{2500}}
  \DeclareUnicodeCharacter{2502}{\sphinxunichar{2502}}
  \DeclareUnicodeCharacter{2514}{\sphinxunichar{2514}}
  \DeclareUnicodeCharacter{251C}{\sphinxunichar{251C}}
  \DeclareUnicodeCharacter{2572}{\textbackslash}
 \fi
\fi
\usepackage{cmap}
\usepackage[T1]{fontenc}
\usepackage{amsmath,amssymb,amstext}
\usepackage{babel}
\usepackage{times}
\usepackage[Bjarne]{fncychap}
\usepackage{sphinx}

\usepackage{geometry}

% Include hyperref last.
\usepackage{hyperref}
% Fix anchor placement for figures with captions.
\usepackage{hypcap}% it must be loaded after hyperref.
% Set up styles of URL: it should be placed after hyperref.
\urlstyle{same}
\addto\captionsenglish{\renewcommand{\contentsname}{Contents:}}

\addto\captionsenglish{\renewcommand{\figurename}{Fig.}}
\addto\captionsenglish{\renewcommand{\tablename}{Table}}
\addto\captionsenglish{\renewcommand{\literalblockname}{Listing}}

\addto\captionsenglish{\renewcommand{\literalblockcontinuedname}{continued from previous page}}
\addto\captionsenglish{\renewcommand{\literalblockcontinuesname}{continues on next page}}

\addto\extrasenglish{\def\pageautorefname{page}}

\setcounter{tocdepth}{1}



\title{rpb-doc-internal Documentation}
\date{Apr 23, 2018}
\release{1}
\author{Wahyu Wijaya Hadiwikarta}
\newcommand{\sphinxlogo}{\vbox{}}
\renewcommand{\releasename}{Release}
\makeindex

\begin{document}

\maketitle
\sphinxtableofcontents
\phantomsection\label{\detokenize{index::doc}}



\chapter{OpenClinica}
\label{\detokenize{trl1:welcome-to-rpb-doc-internal-s-documentation}}\label{\detokenize{trl1::doc}}\label{\detokenize{trl1:openclinica}}
This article documents the installation of OpenClinica. OpenClinica is an open source software to create electronic CRF (Case Report Form). In RadPlanBio, it is one of the main components.

To begin with, login as root (\sphinxcode{\sphinxupquote{sudo su}}). Perform \sphinxcode{\sphinxupquote{yum update}}. Install some support packages such as yum-plugin-remove-with-leaves, links, bash-completion, net-tools, unzip, wget, vim, mlocate, epel-release, lsof, bzip2 and gcc.

Prior to installation, create an install folder to store the required .rpm and .war files.

\fvset{hllines={, ,}}%
\begin{sphinxVerbatim}[commandchars=\\\{\}]
mkdir /usr/local/oc
mkdir /usr/local/oc/install
\end{sphinxVerbatim}


\section{Configurations for OpenClinica}
\label{\detokenize{trl1:configurations-for-openclinica}}

\begin{savenotes}\sphinxattablestart
\centering
\begin{tabulary}{\linewidth}[t]{|T|T|T|T|}
\hline
\sphinxstyletheadfamily 
OS
&\sphinxstyletheadfamily 
Init
&\sphinxstyletheadfamily 
Container, JDK
&\sphinxstyletheadfamily 
Database
\\
\hline
CentOS 7
&
\sphinxcode{\sphinxupquote{systemd}}
&
Tomcat 7, JDK 7
&
PostgreSQL 8.4
\\
\hline
\end{tabulary}
\par
\sphinxattableend\end{savenotes}


\section{List of required files}
\label{\detokenize{trl1:list-of-required-files}}

\begin{savenotes}\sphinxattablestart
\centering
\begin{tabulary}{\linewidth}[t]{|T|T|}
\hline

Tomcat 7
&
apache-tomcat-7.0.84.tar.gz
\\
\hline
JDK 7
&
jdk-7u80-linux-x64.rpm
\\
\hline
PostgreSQL 8.4
&
postgresql84-8.4.22-1PGDG.rhel6.x86\_64.rpm

postgresql84-libs-8.4.22-1PGDG.rhel6.x86\_64.rpm

postgresql84-server-8.4.22-1PGDG.rhel6.x86\_64.rpm

postgresql84-contrib-8.4.22-1PGDG.rhel6.x86\_64.rpm

postgresql84-docs-8.4.22-1PGDG.rhel6.x86\_64.rpm
\\
\hline
\end{tabulary}
\par
\sphinxattableend\end{savenotes}


\section{JDK}
\label{\detokenize{trl1:jdk}}
Remove OpenJDK if originally instaled on the system. Install Oracle jdk

\fvset{hllines={, ,}}%
\begin{sphinxVerbatim}[commandchars=\\\{\}]
\PYG{n+nb}{cd} /usr/local/oc/install
yum install jdk\PYGZhy{}7u80\PYGZhy{}linux\PYGZhy{}x64.rpm
ln \PYGZhy{}s /usr/java/default /usr/local/java
\end{sphinxVerbatim}


\section{Container}
\label{\detokenize{trl1:container}}
Install container Tomcat 7

\fvset{hllines={, ,}}%
\begin{sphinxVerbatim}[commandchars=\\\{\}]
\PYG{n+nb}{cd} /usr/local/oc/install
mkdir /usr/tomcat
tar xvzf apache\PYGZhy{}tomcat\PYGZhy{}7.0.84.tar.gz
mv apache\PYGZhy{}tomcat\PYGZhy{}7.0.84 /usr/tomcat/.
ln \PYGZhy{}s /usr/tomcat/apache\PYGZhy{}tomcat\PYGZhy{}7.0.84 /usr/tomcat/latest
ln \PYGZhy{}s /usr/tomcat/latest /usr/tomcat/default
ln \PYGZhy{}s /usr/tomcat/default /usr/local/tomcat
\end{sphinxVerbatim}

Create user for tomcat service

\fvset{hllines={, ,}}%
\begin{sphinxVerbatim}[commandchars=\\\{\}]
groupadd tomcat1
useradd \PYGZhy{}g tomcat1 tomcat1
chown \PYGZhy{}Rf tomcat1.tomcat1 /usr/tomcat/apache\PYGZhy{}tomcat\PYGZhy{}7.0.84
chown \PYGZhy{}Rf tomcat1.tomcat1 /usr/tomcat/latest/
chown \PYGZhy{}Rf tomcat1.tomcat1 /usr/tomcat/default/
chown \PYGZhy{}Rf tomcat1.tomcat1 /usr/local/tomcat/
\end{sphinxVerbatim}

Create .service file for init

\fvset{hllines={, ,}}%
\begin{sphinxVerbatim}[commandchars=\\\{\}]
vim /lib/systemd/system/tomcat.service
\end{sphinxVerbatim}

\fvset{hllines={, ,}}%
\begin{sphinxVerbatim}[commandchars=\\\{\}]
\PYG{o}{[}Unit\PYG{o}{]}
\PYG{n+nv}{Description}\PYG{o}{=}Tomcat version \PYG{l+m}{7}.0.85
\PYG{n+nv}{Documentation}\PYG{o}{=}https://tomcat.apache.org/download\PYGZhy{}70.cgi
\PYG{n+nv}{After}\PYG{o}{=}syslog.target
\PYG{n+nv}{After}\PYG{o}{=}network.target

\PYG{o}{[}Service\PYG{o}{]}
\PYG{n+nv}{Type}\PYG{o}{=}forking
\PYG{n+nv}{Restart}\PYG{o}{=}always

\PYG{n+nv}{User}\PYG{o}{=}tomcat1
\PYG{n+nv}{Group}\PYG{o}{=}tomcat1

\PYG{n+nv}{Environment}\PYG{o}{=}\PYG{n+nv}{JAVA\PYGZus{}HOME}\PYG{o}{=}/usr/local/java
\PYG{n+nv}{Environment}\PYG{o}{=}\PYG{n+nv}{CATALINA\PYGZus{}HOME}\PYG{o}{=}/usr/local/tomcat
\PYG{n+nv}{Environment}\PYG{o}{=}\PYG{l+s+s1}{\PYGZsq{}JAVA\PYGZus{}OPTS=\PYGZhy{}Xms128m \PYGZhy{}Xmx512m \PYGZhy{}XX:PermSize=128m\PYGZsq{}}

\PYG{n+nv}{ExecStart}\PYG{o}{=}/usr/local/tomcat/bin/startup.sh
\PYG{n+nv}{ExecStop}\PYG{o}{=}/usr/local/tomcat/bin/shutdown.sh
\PYG{n+nv}{SuccessExitStatus}\PYG{o}{=}\PYG{l+m}{143}

\PYG{n+nv}{TimeoutSec}\PYG{o}{=}\PYG{l+m}{0}

\PYG{o}{[}Install\PYG{o}{]}
\PYG{n+nv}{WantedBy}\PYG{o}{=}multi\PYGZhy{}user.target
\end{sphinxVerbatim}

Enable and start the service

\fvset{hllines={, ,}}%
\begin{sphinxVerbatim}[commandchars=\\\{\}]
systemctl \PYG{n+nb}{enable} tomcat.service
systemctl start tomcat.service
systemctl status tomcat.service
\end{sphinxVerbatim}


\section{Database}
\label{\detokenize{trl1:database}}
To install PostgreSQL 8.4, the rpms are needed to be downloaded first

\fvset{hllines={, ,}}%
\begin{sphinxVerbatim}[commandchars=\\\{\}]
\PYG{n+nb}{cd} /usr/local/oc/install/
wget https://yum.postgresql.org/8.4/redhat/rhel\PYGZhy{}6\PYGZhy{}x86\PYGZus{}64/postgresql84\PYGZhy{}libs\PYGZhy{}8.4.22\PYGZhy{}1PGDG.rhel6.x86\PYGZus{}64.rpm
wget https://yum.postgresql.org/8.4/redhat/rhel\PYGZhy{}6\PYGZhy{}x86\PYGZus{}64/postgresql84\PYGZhy{}8.4.22\PYGZhy{}1PGDG.rhel6.x86\PYGZus{}64.rpm
wget https://yum.postgresql.org/8.4/redhat/rhel\PYGZhy{}6\PYGZhy{}x86\PYGZus{}64/postgresql84\PYGZhy{}server\PYGZhy{}8.4.22\PYGZhy{}1PGDG.rhel6.x86\PYGZus{}64.rpm
wget https://yum.postgresql.org/8.4/redhat/rhel\PYGZhy{}6\PYGZhy{}x86\PYGZus{}64/postgresql84\PYGZhy{}contrib\PYGZhy{}8.4.22\PYGZhy{}1PGDG.rhel6.x86\PYGZus{}64.rpm
wget https://yum.postgresql.org/8.4/redhat/rhel\PYGZhy{}6\PYGZhy{}x86\PYGZus{}64/postgresql84\PYGZhy{}docs\PYGZhy{}8.4.22\PYGZhy{}1PGDG.rhel6.x86\PYGZus{}64.rpm
\end{sphinxVerbatim}

Install the rpms

\fvset{hllines={, ,}}%
\begin{sphinxVerbatim}[commandchars=\\\{\}]
yum install postgresql84\PYGZhy{}libs\PYGZhy{}8.4.22\PYGZhy{}1PGDG.rhel6.x86\PYGZus{}64.rpm
yum install postgresql84\PYGZhy{}8.4.22\PYGZhy{}1PGDG.rhel6.x86\PYGZus{}64.rpm
yum install postgresql84\PYGZhy{}server\PYGZhy{}8.4.22\PYGZhy{}1PGDG.rhel6.x86\PYGZus{}64.rpm
yum install uuid
yum install postgresql84\PYGZhy{}contrib\PYGZhy{}8.4.22\PYGZhy{}1PGDG.rhel6.x86\PYGZus{}64.rpm
yum install postgresql84\PYGZhy{}docs\PYGZhy{}8.4.22\PYGZhy{}1PGDG.rhel6.x86\PYGZus{}64.rpm
\end{sphinxVerbatim}

Initialize database for postgres

\fvset{hllines={, ,}}%
\begin{sphinxVerbatim}[commandchars=\\\{\}]
/etc/init.d/postgresql\PYGZhy{}8.4 initdb
\end{sphinxVerbatim}

Edit \sphinxcode{\sphinxupquote{postgresql.conf}}

\fvset{hllines={, ,}}%
\begin{sphinxVerbatim}[commandchars=\\\{\}]
cp /var/lib/pgsql/8.4/data/postgresql.conf /var/lib/pgsql/8.4/data/postgresql.conf.BAK
vim /var/lib/pgsql/8.4/data/postgresql.conf
\end{sphinxVerbatim}

\fvset{hllines={, ,}}%
\begin{sphinxVerbatim}[commandchars=\\\{\}]
listen \PYG{n+nv}{addresses} \PYG{o}{=} \PYG{l+s+s1}{\PYGZsq{}*\PYGZsq{}}
\PYG{n+nv}{port} \PYG{o}{=} \PYG{l+m}{5432}
\end{sphinxVerbatim}

Create symlink to \sphinxcode{\sphinxupquote{/opt/PostgreSQL}}

\fvset{hllines={, ,}}%
\begin{sphinxVerbatim}[commandchars=\\\{\}]
mkdir /opt/PostgreSQL
ln \PYGZhy{}s /usr/pgsql\PYGZhy{}8.4 /opt/PostgreSQL/8.4
ln \PYGZhy{}s /var/lib/pgsql/8.4/data /opt/PostgreSQL/8.4/data
\end{sphinxVerbatim}

Prior to running the database, create an init .service file (non-optimized)

\fvset{hllines={, ,}}%
\begin{sphinxVerbatim}[commandchars=\\\{\}]
rm /etc/init.d/postgresql\PYGZhy{}8.4
vim /lib/systemd/system/postgresql\PYGZhy{}8.4.service
\end{sphinxVerbatim}

\fvset{hllines={, ,}}%
\begin{sphinxVerbatim}[commandchars=\\\{\}]
\PYG{o}{[}Unit\PYG{o}{]}
\PYG{n+nv}{Description}\PYG{o}{=}PostgreSQL \PYG{l+m}{8}.4 database server
\PYG{n+nv}{Documentation}\PYG{o}{=}https://www.postgresql.org/docs/8.4/static/index.html
\PYG{n+nv}{After}\PYG{o}{=}syslog.target
\PYG{n+nv}{After}\PYG{o}{=}network.target

\PYG{o}{[}Service\PYG{o}{]}
\PYG{n+nv}{Type}\PYG{o}{=}forking
\PYG{n+nv}{Restart}\PYG{o}{=}always

\PYG{n+nv}{User}\PYG{o}{=}postgres
\PYG{n+nv}{Group}\PYG{o}{=}postgres

\PYG{n+nv}{OOMScoreAdjust}\PYG{o}{=}\PYGZhy{}1000
\PYG{n+nv}{Environment}\PYG{o}{=}\PYG{n+nv}{PG\PYGZus{}OOM\PYGZus{}ADJUST\PYGZus{}FILE}\PYG{o}{=}/proc/self/oom\PYGZus{}score\PYGZus{}adj
\PYG{n+nv}{Environment}\PYG{o}{=}\PYG{n+nv}{PG\PYGZus{}OOM\PYGZus{}ADJUST\PYGZus{}VALUE}\PYG{o}{=}\PYG{l+m}{0}

\PYG{n+nv}{Environment}\PYG{o}{=}\PYG{n+nv}{PGDATA}\PYG{o}{=}/var/lib/pgsql/8.4/data/

\PYG{n+nv}{ExecStart}\PYG{o}{=}/usr/pgsql\PYGZhy{}8.4/bin/pg\PYGZus{}ctl start \PYGZhy{}D \PYG{l+s+si}{\PYGZdl{}\PYGZob{}}\PYG{n+nv}{PGDATA}\PYG{l+s+si}{\PYGZcb{}} \PYGZhy{}s \PYGZhy{}w \PYGZhy{}t \PYG{l+m}{300}
\PYG{n+nv}{ExecStop}\PYG{o}{=}/usr/pgsql\PYGZhy{}8.4/bin/pg\PYGZus{}ctl stop \PYGZhy{}D \PYG{l+s+si}{\PYGZdl{}\PYGZob{}}\PYG{n+nv}{PGDATA}\PYG{l+s+si}{\PYGZcb{}} \PYGZhy{}s \PYGZhy{}m fast
\PYG{n+nv}{ExecReload}\PYG{o}{=}/usr/pgsql\PYGZhy{}8.4/bin/pg\PYGZus{}ctl reload \PYGZhy{}D \PYG{l+s+si}{\PYGZdl{}\PYGZob{}}\PYG{n+nv}{PGDATA}\PYG{l+s+si}{\PYGZcb{}} \PYGZhy{}s

\PYG{n+nv}{TimeoutSec}\PYG{o}{=}\PYG{l+m}{0}

\PYG{o}{[}Install\PYG{o}{]}
\PYG{n+nv}{WantedBy}\PYG{o}{=}multi\PYGZhy{}user.target
\end{sphinxVerbatim}

Enable and start the service

\fvset{hllines={, ,}}%
\begin{sphinxVerbatim}[commandchars=\\\{\}]
systemctl \PYG{n+nb}{enable} postgresql\PYGZhy{}8.4.service
systemctl start postgresql\PYGZhy{}8.4.service
systemctl status postgresql\PYGZhy{}8.4.service
\end{sphinxVerbatim}

Assign password to user postgres

\fvset{hllines={, ,}}%
\begin{sphinxVerbatim}[commandchars=\\\{\}]
su postgres
psql

ALTER USER postgres PASSWORD \PYG{l+s+s1}{\PYGZsq{}xxxxxxxxx\PYGZsq{}}\PYG{p}{;}
\PYG{l+s+se}{\PYGZbs{}q}

\PYG{n+nb}{exit}
\end{sphinxVerbatim}

Edit authentication method to access database by editing \sphinxcode{\sphinxupquote{pg\_hba.conf}}

\fvset{hllines={, ,}}%
\begin{sphinxVerbatim}[commandchars=\\\{\}]
cp /var/lib/pgsql/8.4/data/pg\PYGZus{}hba.conf /var/lib/pgsql/8.4/data/pg\PYGZus{}hba.conf.BAK
vim /var/lib/pgsql/8.4/data/pg\PYGZus{}hba.conf
\end{sphinxVerbatim}

\fvset{hllines={, ,}}%
\begin{sphinxVerbatim}[commandchars=\\\{\}]
\PYG{c+c1}{\PYGZsh{} TYPE  DATABASE        USER            ADDRESS                 METHOD}
\PYG{n+nb}{local}   all             all                                     md5
host    all             all             \PYG{l+m}{127}.0.0.1/32            md5
\end{sphinxVerbatim}

Setup database for OpenClinica

\fvset{hllines={, ,}}%
\begin{sphinxVerbatim}[commandchars=\\\{\}]
psql \PYGZhy{}U postgres \PYGZhy{}c \PYG{l+s+s2}{\PYGZdq{}CREATE ROLE clinica LOGIN ENCRYPTED PASSWORD \PYGZsq{}xxxxxxxxx\PYGZsq{} SUPERUSER NOINHERIT NOCREATEDB NOCREATEROLE\PYGZdq{}}
psql \PYGZhy{}U postgres \PYGZhy{}c \PYG{l+s+s2}{\PYGZdq{}CREATE DATABASE openclinica WITH ENCODING=\PYGZsq{}UTF8\PYGZsq{} OWNER=clinica\PYGZdq{}}
psql \PYGZhy{}U postgres

ALTER USER clinica WITH PASSWORD \PYG{l+s+s1}{\PYGZsq{}xxxxxxxxx\PYGZsq{}}\PYG{p}{;}
\PYG{l+s+se}{\PYGZbs{}q}
\end{sphinxVerbatim}


\section{OpenClinica}
\label{\detokenize{trl1:id1}}
We go to the install folder and unzip OpenClinica war file

\fvset{hllines={, ,}}%
\begin{sphinxVerbatim}[commandchars=\\\{\}]
systemctl stop tomcat.service
\end{sphinxVerbatim}

\fvset{hllines={, ,}}%
\begin{sphinxVerbatim}[commandchars=\\\{\}]
\PYG{n+nb}{cd} /usr/local/oc/install
unzip OpenClinica.war \PYGZhy{}d OpenClinica
mv OpenClinica /usr/local/tomcat/webapps
\end{sphinxVerbatim}

Edit \sphinxcode{\sphinxupquote{datainfo.properties}} file

\fvset{hllines={, ,}}%
\begin{sphinxVerbatim}[commandchars=\\\{\}]
vim /usr/local/tomcat/webapps/OpenClinica/WEB\PYGZhy{}INF/classes/datainfo.properties
\end{sphinxVerbatim}

\fvset{hllines={, ,}}%
\begin{sphinxVerbatim}[commandchars=\\\{\}]
\PYG{n+nv}{dbType}\PYG{o}{=}postgres
\PYG{n+nv}{dbUser}\PYG{o}{=}clinica
\PYG{n+nv}{dbPass}\PYG{o}{=}xxxxxxxxx
\PYG{n+nv}{db}\PYG{o}{=}openclinica
\PYG{n+nv}{dbPort}\PYG{o}{=}\PYG{l+m}{5432}
\PYG{n+nv}{dbHost}\PYG{o}{=}localhost

\PYG{n+nv}{filePath}\PYG{o}{=}\PYG{l+s+si}{\PYGZdl{}\PYGZob{}}\PYG{n+nv}{catalina}\PYG{p}{.home}\PYG{l+s+si}{\PYGZcb{}}/\PYG{l+s+si}{\PYGZdl{}\PYGZob{}}\PYG{n+nv}{WEBAPP}\PYG{p}{.lower}\PYG{l+s+si}{\PYGZcb{}}.data/

\PYG{n+nv}{sysURL}\PYG{o}{=}http://localhost:8080/\PYG{l+s+si}{\PYGZdl{}\PYGZob{}}\PYG{n+nv}{WEBAPP}\PYG{l+s+si}{\PYGZcb{}}/MainMenu

log.dir\PYG{o}{=}\PYG{l+s+si}{\PYGZdl{}\PYGZob{}}\PYG{n+nv}{catalina}\PYG{p}{.home}\PYG{l+s+si}{\PYGZcb{}}/logs/openclinica
\PYG{n+nv}{logLocation}\PYG{o}{=}\PYG{n+nb}{local}

\PYG{n+nv}{logLevel}\PYG{o}{=}info
syslog.host\PYG{o}{=}localhost
syslog.port\PYG{o}{=}\PYG{l+m}{514}
\end{sphinxVerbatim}

Start tomcat and depends on the \sphinxcode{\sphinxupquote{filePath}} parameter, a folder will be created that contains a new \sphinxcode{\sphinxupquote{datainfo.properties}}. Subsequent changes on the settings must be performed on this file.

\fvset{hllines={, ,}}%
\begin{sphinxVerbatim}[commandchars=\\\{\}]
systemctl start tomcat.service
\end{sphinxVerbatim}

Test connection to OpenClinica

\fvset{hllines={, ,}}%
\begin{sphinxVerbatim}[commandchars=\\\{\}]
links \PYG{l+m}{127}.0.0.1:8080/OpenClinica/MainMenu
\end{sphinxVerbatim}


\section{OpenClinica Web Service}
\label{\detokenize{trl1:openclinica-web-service}}
The procedure is the same like installing OpenClinica

We go to the install folder and unzip OpenClinica war file

\fvset{hllines={, ,}}%
\begin{sphinxVerbatim}[commandchars=\\\{\}]
systemctl stop tomcat.service
\end{sphinxVerbatim}

\fvset{hllines={, ,}}%
\begin{sphinxVerbatim}[commandchars=\\\{\}]
\PYG{n+nb}{cd} /usr/local/oc/install
unzip OpenClinica\PYGZhy{}ws.war \PYGZhy{}d OpenClinica\PYGZhy{}ws
mv OpenClinica\PYGZhy{}ws /usr/local/tomcat/webapps
\end{sphinxVerbatim}

Edit \sphinxcode{\sphinxupquote{datainfo.properties}} file

\fvset{hllines={, ,}}%
\begin{sphinxVerbatim}[commandchars=\\\{\}]
vim /usr/local/tomcat/webapps/OpenClinica\PYGZhy{}ws/WEB\PYGZhy{}INF/classes/datainfo.properties
\end{sphinxVerbatim}

\fvset{hllines={, ,}}%
\begin{sphinxVerbatim}[commandchars=\\\{\}]
\PYG{n+nv}{dbType}\PYG{o}{=}postgres
\PYG{n+nv}{dbUser}\PYG{o}{=}clinica
\PYG{n+nv}{dbPass}\PYG{o}{=}xxxxxxxxx
\PYG{n+nv}{db}\PYG{o}{=}openclinica
\PYG{n+nv}{dbPort}\PYG{o}{=}\PYG{l+m}{5432}
\PYG{n+nv}{dbHost}\PYG{o}{=}localhost

\PYG{n+nv}{filePath}\PYG{o}{=}\PYG{l+s+si}{\PYGZdl{}\PYGZob{}}\PYG{n+nv}{catalina}\PYG{p}{.home}\PYG{l+s+si}{\PYGZcb{}}/\PYG{l+s+si}{\PYGZdl{}\PYGZob{}}\PYG{n+nv}{WEBAPP}\PYG{p}{.lower}\PYG{l+s+si}{\PYGZcb{}}.data/

\PYG{n+nv}{sysURL}\PYG{o}{=}http://localhost:8080/\PYG{l+s+si}{\PYGZdl{}\PYGZob{}}\PYG{n+nv}{WEBAPP}\PYG{l+s+si}{\PYGZcb{}}/MainMenu

log.dir\PYG{o}{=}\PYG{l+s+si}{\PYGZdl{}\PYGZob{}}\PYG{n+nv}{catalina}\PYG{p}{.home}\PYG{l+s+si}{\PYGZcb{}}/logs/openclinica
\PYG{n+nv}{logLocation}\PYG{o}{=}\PYG{n+nb}{local}

\PYG{n+nv}{logLevel}\PYG{o}{=}info
syslog.host\PYG{o}{=}localhost
syslog.port\PYG{o}{=}\PYG{l+m}{514}
\end{sphinxVerbatim}

Start Tomcat and another folder will be created too with a new \sphinxcode{\sphinxupquote{datainfo.properties}} file.

\fvset{hllines={, ,}}%
\begin{sphinxVerbatim}[commandchars=\\\{\}]
systemctl start tomcat.service
\end{sphinxVerbatim}

Test connection to OpenClinica Web Service

\fvset{hllines={, ,}}%
\begin{sphinxVerbatim}[commandchars=\\\{\}]
links \PYG{l+m}{127}.0.0.1:8080/OpenClinica\PYGZhy{}ws/MainMenu
\end{sphinxVerbatim}


\chapter{PACS Conquest}
\label{\detokenize{trl2:pacs-conquest}}\label{\detokenize{trl2::doc}}
This article documents the installation of PACS Conquest. Conquest is an open source project for PACS. In the RadPlanBio project, Conquest is used as the DICOM server component.

To begin with, login as root (\sphinxcode{\sphinxupquote{sudo su}}). Perform \sphinxcode{\sphinxupquote{yum update}}. Install some support packages such as yum-plugin-remove-with-leaves, links, bash-completion, net-tools, unzip, wget, vim, mlocate, epel-release, lsof, bzip2 and gcc.


\section{Configurations for DICOM server}
\label{\detokenize{trl2:configurations-for-dicom-server}}

\begin{savenotes}\sphinxattablestart
\centering
\begin{tabulary}{\linewidth}[t]{|T|T|T|T|}
\hline
\sphinxstyletheadfamily 
OS
&\sphinxstyletheadfamily 
Init
&\sphinxstyletheadfamily 
Server
&\sphinxstyletheadfamily 
Database
\\
\hline
CentOS 7
&
\sphinxcode{\sphinxupquote{systemd}}
&
Apache2
&
PostgreSQL 9.5
\\
\hline
\end{tabulary}
\par
\sphinxattableend\end{savenotes}


\section{Database}
\label{\detokenize{trl2:database}}
For database PostgreSQL 9.5, the installation steps is as follows

\fvset{hllines={, ,}}%
\begin{sphinxVerbatim}[commandchars=\\\{\}]
yum install https://download.postgresql.org/pub/repos/yum/9.5/redhat/rhel\PYGZhy{}7\PYGZhy{}x86\PYGZus{}64/pgdg\PYGZhy{}centos95\PYGZhy{}9.5\PYGZhy{}3.noarch.rpm
yum install postgresql95 postgresql95\PYGZhy{}server postgresql95\PYGZhy{}contrib postgresql95\PYGZhy{}docs
/usr/pgsql\PYGZhy{}9.5/bin/postgresql95\PYGZhy{}setup initdb
\end{sphinxVerbatim}

Edit \sphinxcode{\sphinxupquote{postgresql.conf}}

\fvset{hllines={, ,}}%
\begin{sphinxVerbatim}[commandchars=\\\{\}]
cp /var/lib/pgsql/9.5/data/postgresql.conf /var/lib/pgsql/9.5/data/postgresql.conf.BAK
vim /var/lib/pgsql/9.5/data/postgresql.conf
\end{sphinxVerbatim}

\fvset{hllines={, ,}}%
\begin{sphinxVerbatim}[commandchars=\\\{\}]
listen \PYG{n+nv}{addresses} \PYG{o}{=} \PYG{l+s+s1}{\PYGZsq{}*\PYGZsq{}}
\PYG{n+nv}{port} \PYG{o}{=} \PYG{l+m}{5432}
\end{sphinxVerbatim}

Enable init script

\fvset{hllines={, ,}}%
\begin{sphinxVerbatim}[commandchars=\\\{\}]
systemctl \PYG{n+nb}{enable} postgresql\PYGZhy{}9.5.service
systemctl start postgresql\PYGZhy{}9.5
systemctl status postgresql\PYGZhy{}9.5.service
\end{sphinxVerbatim}

Create symlink to \sphinxcode{\sphinxupquote{/opt/PostgreSQL}}

\fvset{hllines={, ,}}%
\begin{sphinxVerbatim}[commandchars=\\\{\}]
mkdir /opt/PostgreSQL
ln \PYGZhy{}s /usr/pgsql\PYGZhy{}9.5 /opt/PostgreSQL/9.5
ln \PYGZhy{}s /var/lib/pgsql/9.5/data /opt/PostgreSQL/9.5/data
\end{sphinxVerbatim}

Assign password to user postgres

\fvset{hllines={, ,}}%
\begin{sphinxVerbatim}[commandchars=\\\{\}]
su postgres
psql

ALTER USER postgres PASSWORD \PYG{l+s+s1}{\PYGZsq{}xxxxxxxxx\PYGZsq{}}\PYG{p}{;}
\PYG{l+s+se}{\PYGZbs{}q}

\PYG{n+nb}{exit}
\end{sphinxVerbatim}

Edit authentication method to access database by editing \sphinxcode{\sphinxupquote{pg\_hba.conf}}

\fvset{hllines={, ,}}%
\begin{sphinxVerbatim}[commandchars=\\\{\}]
cp /var/lib/pgsql/9.5/data/pg\PYGZus{}hba.conf /var/lib/pgsql/9.5/data/pg\PYGZus{}hba.conf.BAK
vim /var/lib/pgsql/9.5/data/pg\PYGZus{}hba.conf
\end{sphinxVerbatim}

\fvset{hllines={, ,}}%
\begin{sphinxVerbatim}[commandchars=\\\{\}]
\PYG{c+c1}{\PYGZsh{} TYPE  DATABASE        USER            ADDRESS                 METHOD}
\PYG{n+nb}{local}   all             all                                     md5
host    all             all             \PYG{l+m}{127}.0.0.1/32            md5
\end{sphinxVerbatim}

Setup the database for Conquest

\fvset{hllines={, ,}}%
\begin{sphinxVerbatim}[commandchars=\\\{\}]
psql \PYGZhy{}U postgres \PYGZhy{}c \PYG{l+s+s2}{\PYGZdq{}CREATE ROLE conquest LOGIN ENCRYPTED PASSWORD \PYGZsq{}xxxxxxxxx\PYGZsq{} SUPERUSER NOINHERIT NOCREATEDB NOCREATEROLE\PYGZdq{}}
psql \PYGZhy{}U postgres \PYGZhy{}c \PYG{l+s+s2}{\PYGZdq{}CREATE DATABASE conquest WITH ENCODING=\PYGZsq{}UTF8\PYGZsq{} OWNER=conquest\PYGZdq{}}
psql \PYGZhy{}U postgres

ALTER USER conquest WITH PASSWORD \PYG{l+s+s1}{\PYGZsq{}xxxxxxxxx\PYGZsq{}}\PYG{p}{;}
\PYG{l+s+se}{\PYGZbs{}q}
\end{sphinxVerbatim}

Restart PostgreSQL 9.5 service

\fvset{hllines={, ,}}%
\begin{sphinxVerbatim}[commandchars=\\\{\}]
systemctl restart postgresql\PYGZhy{}9.5.service
systemctl status postgresql\PYGZhy{}9.5.service
\end{sphinxVerbatim}


\section{Web Server}
\label{\detokenize{trl2:web-server}}
Install Apache2 via yum

\fvset{hllines={, ,}}%
\begin{sphinxVerbatim}[commandchars=\\\{\}]
yum install httpd
rm \PYGZhy{}f /etc/httpd/conf.d/welcome.conf
\end{sphinxVerbatim}

Edit \sphinxcode{\sphinxupquote{httpd.conf}}

\fvset{hllines={, ,}}%
\begin{sphinxVerbatim}[commandchars=\\\{\}]
cp /etc/httpd/conf/httpd.conf /etc/httpd/conf/httpd.conf.BAK
vim /etc/httpd/conf/httpd.conf
\end{sphinxVerbatim}

\fvset{hllines={, ,}}%
\begin{sphinxVerbatim}[commandchars=\\\{\}]
ServerAdmin admin@domain.de
ServerName hostname.domain.de:80

\PYGZlt{}IfModule dir\PYGZus{}module\PYGZgt{}
  DirectoryIndex index.html index.htm
\PYGZlt{}/IfModule\PYGZgt{}
\end{sphinxVerbatim}

Test syntax and enable init service

\fvset{hllines={, ,}}%
\begin{sphinxVerbatim}[commandchars=\\\{\}]
apachectl configtest
systemctl \PYG{n+nb}{enable} httpd.service
systemctl start httpd.service
systemctl status httpd.service
\end{sphinxVerbatim}

Enable CGI and manage SELinux*

\fvset{hllines={, ,}}%
\begin{sphinxVerbatim}[commandchars=\\\{\}]
yum install policycoreutils\PYGZhy{}python
setsebool \PYGZhy{}P httpd\PYGZus{}enable\PYGZus{}cgi \PYG{l+m}{1}
semanage fcontext \PYGZhy{}a \PYGZhy{}t httpd\PYGZus{}sys\PYGZus{}script\PYGZus{}exec\PYGZus{}t /var/www/cgi\PYGZhy{}bin
restorecon \PYGZhy{}Rv /var/www/cgi\PYGZhy{}bin
systemctl restart httpd.service
systemctl status httpd.service
\end{sphinxVerbatim}

Additionally if wanting to test CGI with perl or ruby script

\fvset{hllines={, ,}}%
\begin{sphinxVerbatim}[commandchars=\\\{\}]
yum install perl perl\PYGZhy{}CGI ruby
\end{sphinxVerbatim}


\section{Compiling the DICOM server}
\label{\detokenize{trl2:compiling-the-dicom-server}}
Start by installing required packages

\fvset{hllines={, ,}}%
\begin{sphinxVerbatim}[commandchars=\\\{\}]
yum install gcc\PYGZhy{}c++ gcc\PYGZhy{}c++\PYGZhy{}sh\PYGZhy{}linux\PYGZhy{}gnu clang
\end{sphinxVerbatim}

Create an install folder

\fvset{hllines={, ,}}%
\begin{sphinxVerbatim}[commandchars=\\\{\}]
mkdir /usr/local/pacs/install
\end{sphinxVerbatim}

Download dicomserver1419b.zip from \sphinxurl{https://ingenium.home.xs4all.nl/dicom.html} and save to install folder.

Unzip the zip file

\fvset{hllines={, ,}}%
\begin{sphinxVerbatim}[commandchars=\\\{\}]
unzip /usr/local/pacs/install/dicomserver1419b.zip \PYGZhy{}d /usr/local/pacs/dicomserver1419b
\end{sphinxVerbatim}

Create symlink to \sphinxcode{\sphinxupquote{/opt}}

\fvset{hllines={, ,}}%
\begin{sphinxVerbatim}[commandchars=\\\{\}]
ln \PYGZhy{}s /usr/local/pacs/dicomserver1419b/distribution /opt/conquest\PYGZhy{}14\PYGZhy{}19b
\end{sphinxVerbatim}

Create folder for incoming DICOM data

\fvset{hllines={, ,}}%
\begin{sphinxVerbatim}[commandchars=\\\{\}]
mkdir /opt/conquest\PYGZhy{}14\PYGZhy{}19b/data/incoming
\end{sphinxVerbatim}

Create user and group \sphinxcode{\sphinxupquote{conquest}}

\fvset{hllines={, ,}}%
\begin{sphinxVerbatim}[commandchars=\\\{\}]
groupadd conquest
useradd \PYGZhy{}g conquest conquest
\end{sphinxVerbatim}

Remove shell login for user \sphinxcode{\sphinxupquote{conquest}}

\fvset{hllines={, ,}}%
\begin{sphinxVerbatim}[commandchars=\\\{\}]
vim /etc/passwd
\end{sphinxVerbatim}

\fvset{hllines={, ,}}%
\begin{sphinxVerbatim}[commandchars=\\\{\}]
conquest:x:1000:1000::/home/conquest:/bin/false
\end{sphinxVerbatim}

Change ownership and set permission

\fvset{hllines={, ,}}%
\begin{sphinxVerbatim}[commandchars=\\\{\}]
chown \PYGZhy{}R conquest:conquest /usr/local/pacs/dicomserver1419b
chown \PYGZhy{}R conquest:conquest /opt/conquest\PYGZhy{}14\PYGZhy{}19b
chmod g+w /opt/conquest\PYGZhy{}14\PYGZhy{}19b/data/incoming
\end{sphinxVerbatim}

Prepare for compiling source code

\fvset{hllines={, ,}}%
\begin{sphinxVerbatim}[commandchars=\\\{\}]
ln \PYGZhy{}s /usr/pgsql\PYGZhy{}9.5/lib/libpq.so.5 /usr/lib/libpq.so
mkdir /usr/local/man
mkdir /usr/local/man/man1
\end{sphinxVerbatim}

Edit the \sphinxcode{\sphinxupquote{total.cpp}} file to move \sphinxcode{\sphinxupquote{aaac.cxx}} on top \sphinxcode{\sphinxupquote{qrsop.cxx}}. Uncomment the line below \sphinxcode{\sphinxupquote{aarj.cxx}} so \sphinxcode{\sphinxupquote{aaac.cxx}} will be read first sequentially and will not be compiled twice. This is because the function min() that is required to compile \sphinxcode{\sphinxupquote{qrsop.cxx}} is defined in \sphinxcode{\sphinxupquote{aaac.cxx}}.

\fvset{hllines={, ,}}%
\begin{sphinxVerbatim}[commandchars=\\\{\}]
\PYG{n+nb}{cd} /opt/conquest\PYGZhy{}14\PYGZhy{}19b/src/dgate/src
vim total.cpp
\end{sphinxVerbatim}

\fvset{hllines={, ,}}%
\begin{sphinxVerbatim}[commandchars=\\\{\}]
\PYG{c+c1}{\PYGZsh{}include \PYGZdq{}aaac.cxx\PYGZdq{}}
\PYG{c+c1}{\PYGZsh{}include \PYGZdq{}qrsop.cxx\PYGZdq{}}

\PYG{c+c1}{\PYGZsh{}include \PYGZdq{}aarj.cxx\PYGZdq{}}
//\PYGZsh{}include \PYG{l+s+s2}{\PYGZdq{}aaac.cxx\PYGZdq{}}
\end{sphinxVerbatim}

Edit/create new compile script

\fvset{hllines={, ,}}%
\begin{sphinxVerbatim}[commandchars=\\\{\}]
\PYG{n+nb}{cd} /opt/conquest\PYGZhy{}14\PYGZhy{}19b
cp maklinux maklinux.BAK
chmod \PYG{l+m}{755} maklinux
vim maklinux
\end{sphinxVerbatim}

\fvset{hllines={, ,}}%
\begin{sphinxVerbatim}[commandchars=\\\{\}]
\PYG{c+ch}{\PYGZsh{}!/bin/bash}

\PYG{n+nv}{SRC}\PYG{o}{=}./src/dgate\PYG{p}{;}
\PYG{n+nv}{CONF}\PYG{o}{=}./linux/conf\PYG{p}{;}
\PYG{n+nv}{LINUX}\PYG{o}{=}./linux\PYG{p}{;}

chmod \PYG{l+m}{777} src/dgate/jpeg\PYGZhy{}6c/configure
\PYG{n+nb}{cd} src/dgate/jpeg\PYGZhy{}6c
./configure
make
sudo make install
\PYG{n+nb}{cd} ../../..

\PYG{n+nb}{export} \PYG{n+nv}{LD\PYGZus{}LIBRARY\PYGZus{}PATH}\PYG{o}{=}\PYG{l+s+s2}{\PYGZdq{}/usr/pgsql\PYGZhy{}9.5/lib/\PYGZdq{}}\PYG{p}{;}
gcc \PYGZhy{}o \PYG{n+nv}{\PYGZdl{}SRC}/lua.o \PYGZhy{}c \PYG{n+nv}{\PYGZdl{}SRC}/lua\PYGZus{}5.1.5/all.c \PYGZhy{}I\PYG{n+nv}{\PYGZdl{}SRC}/lua\PYGZus{}5.1.5 \PYGZhy{}DLUA\PYGZus{}USE\PYGZus{}DLOPEN \PYGZhy{}DLUA\PYGZus{}USE\PYGZus{}POSIX\PYG{p}{;}
g++ \PYGZhy{}std\PYG{o}{=}c++11 \PYGZhy{}o \PYG{n+nv}{\PYGZdl{}SRC}/charls.o \PYGZhy{}c \PYG{n+nv}{\PYGZdl{}SRC}/charls/all.cpp \PYGZhy{}I\PYG{n+nv}{\PYGZdl{}SRC}/charls
gcc \PYGZhy{}o \PYG{n+nv}{\PYGZdl{}SRC}/openjpeg.o \PYGZhy{}c \PYG{n+nv}{\PYGZdl{}SRC}/openjpeg/all.c \PYGZhy{}I\PYG{n+nv}{\PYGZdl{}SRC}/openjpeg

g++ \PYGZhy{}std\PYG{o}{=}c++11 \PYGZhy{}I/usr/pgsql\PYGZhy{}9.5/include/ \PYGZhy{}DUNIX \PYGZhy{}DNATIVE\PYGZus{}ENDIAN\PYG{o}{=}\PYG{l+m}{1} \PYGZhy{}DHAVE\PYGZus{}LIBJPEG \PYGZhy{}DPOSTGRES \PYGZhy{}DHAVE\PYGZus{}LIBCHARLS \PYGZhy{}DHAVE\PYGZus{}LIBOPENJPEG2 \PYGZhy{}Wno\PYGZhy{}write\PYGZhy{}strings \PYG{n+nv}{\PYGZdl{}SRC}/lua.o \PYG{n+nv}{\PYGZdl{}SRC}/charls.o \PYG{n+nv}{\PYGZdl{}SRC}/openjpeg.o \PYGZhy{}o dgate \PYGZhy{}lpthread \PYGZhy{}ldl \PYGZhy{}I\PYG{n+nv}{\PYGZdl{}SRC}/src \PYG{n+nv}{\PYGZdl{}SRC}/src/total.cpp \PYGZhy{}I\PYG{n+nv}{\PYGZdl{}SRC}/dicomlib \PYGZhy{}L/usr/pgsql\PYGZhy{}9.5/lib/ \PYGZhy{}lpq  \PYGZhy{}ljpeg \PYGZhy{}I\PYG{n+nv}{\PYGZdl{}SRC}/jpeg\PYGZhy{}6c \PYGZhy{}L\PYG{n+nv}{\PYGZdl{}SRC}/jpeg\PYGZhy{}6c \PYGZhy{}I\PYG{n+nv}{\PYGZdl{}SRC}/lua\PYGZus{}5.1.5 \PYGZhy{}I\PYG{n+nv}{\PYGZdl{}SRC}/openjpeg \PYGZhy{}I\PYG{n+nv}{\PYGZdl{}SRC}/charls \PYGZhy{}Wno\PYGZhy{}multichar\PYG{p}{;}

rm \PYG{n+nv}{\PYGZdl{}SRC}/lua.o\PYG{p}{;}
rm \PYG{n+nv}{\PYGZdl{}SRC}/charls.o\PYG{p}{;}
rm \PYG{n+nv}{\PYGZdl{}SRC}/openjpeg.o\PYG{p}{;}

pkill \PYGZhy{}9 dgate\PYG{p}{;}
sleep \PYG{l+m}{0}.2s\PYG{p}{;}

cp \PYG{n+nv}{\PYGZdl{}CONF}/dicom.ini.postgres dicom.ini\PYG{p}{;}
cp \PYG{n+nv}{\PYGZdl{}CONF}/dicom.sql.postgres dicom.sql\PYG{p}{;}

cp \PYG{n+nv}{\PYGZdl{}LINUX}/acrnema.map acrnema.map\PYG{p}{;}
cp \PYG{n+nv}{\PYGZdl{}LINUX}/dgatesop.lst dgatesop.lst\PYG{p}{;}

cp /opt/conquest\PYGZhy{}14\PYGZhy{}19b/dgate /var/www/cgi\PYGZhy{}bin/dgate \PYG{p}{;}
cp /opt/conquest\PYGZhy{}14\PYGZhy{}19b/dicom.sql /var/www/cgi\PYGZhy{}bin/dicom.sql \PYG{p}{;}
cp /opt/conquest\PYGZhy{}14\PYGZhy{}19b/acrnema.map /var/www/cgi\PYGZhy{}bin/acrnema.map \PYG{p}{;}

cp \PYGZhy{}r /opt/conquest\PYGZhy{}14\PYGZhy{}19b/webserver/cgi\PYGZhy{}bin/* /var/www/cgi\PYGZhy{}bin\PYG{p}{;}
cp \PYGZhy{}r /opt/conquest\PYGZhy{}14\PYGZhy{}19b/webserver/cgi\PYGZhy{}bin/.lua /var/www/cgi\PYGZhy{}bin\PYG{p}{;}
cp \PYGZhy{}r /opt/conquest\PYGZhy{}14\PYGZhy{}19b/webserver/cgi\PYGZhy{}bin/.lua.linux /var/www/cgi\PYGZhy{}bin\PYG{p}{;}

cp /var/www/cgi\PYGZhy{}bin/dicom.ini.linux /var/www/cgi\PYGZhy{}bin/dicom.ini\PYG{p}{;}
cp /var/www/cgi\PYGZhy{}bin/newweb/dicom.ini.linux /var/www/cgi\PYGZhy{}bin/newweb/dicom.ini\PYG{p}{;}
cp /var/www/cgi\PYGZhy{}bin/.lua.linux /var/www/cgi\PYGZhy{}bin/.lua\PYG{p}{;}

cp /opt/conquest\PYGZhy{}14\PYGZhy{}19b/dgate /var/www/cgi\PYGZhy{}bin/newweb/dgate \PYG{p}{;}
cp /opt/conquest\PYGZhy{}14\PYGZhy{}19b/acrnema.map /var/www/cgi\PYGZhy{}bin/newweb/acrnema.map \PYG{p}{;}

cp \PYGZhy{}r /opt/conquest\PYGZhy{}14\PYGZhy{}19b/webserver/htdocs/* /var/www\PYG{p}{;}
cp \PYGZhy{}r /opt/conquest\PYGZhy{}14\PYGZhy{}19b/webserver/htdocs/* /var/www/html\PYG{p}{;}

mkdir /opt/conquest\PYGZhy{}14\PYGZhy{}19b/logs
chown \PYGZhy{}R conquest:conquest /opt/conquest\PYGZhy{}14\PYGZhy{}19b/
\end{sphinxVerbatim}

Run the compile script

\fvset{hllines={, ,}}%
\begin{sphinxVerbatim}[commandchars=\\\{\}]
/opt/conquest\PYGZhy{}14\PYGZhy{}19b/maklinux
\end{sphinxVerbatim}

If everything is right, on CentOS 7, compilation will run smooth with some warnings that can be ignored.


\section{Setting the DICOM server}
\label{\detokenize{trl2:setting-the-dicom-server}}
After finish with installation of the DICOM server, the next step is to have the correct settings.


\begin{savenotes}\sphinxattablestart
\centering
\begin{tabulary}{\linewidth}[t]{|T|T|}
\hline
\sphinxstyletheadfamily 
File
&\sphinxstyletheadfamily 
Path
\\
\hline
dicom.ini
&
/opt/conquest-14-19b/dicom.ini

/var/www/cgi-bin/dicom.ini

/var/www/cgi-bin/newweb/dicom.ini
\\
\hline
acrnema.map
&
/opt/conquest-14-19b/acrnema.map

/var/www/cgi-bin/acrnema.map

/var/www/cgi-bin/newweb/acrnema.map
\\
\hline
\end{tabulary}
\par
\sphinxattableend\end{savenotes}


\subsection{Edit dicom.ini}
\label{\detokenize{trl2:edit-dicom-ini}}
\fvset{hllines={, ,}}%
\begin{sphinxVerbatim}[commandchars=\\\{\}]
vim /opt/conquest\PYGZhy{}14\PYGZhy{}19b/dicom.ini
\end{sphinxVerbatim}

\fvset{hllines={, ,}}%
\begin{sphinxVerbatim}[commandchars=\\\{\}]
\PYG{o}{[}sscscp\PYG{o}{]}
\PYG{n+nv}{MicroPACS}                \PYG{o}{=} sscscp

\PYG{c+c1}{\PYGZsh{} Network configuration: server name and TCP/IP port\PYGZsh{}}
\PYG{n+nv}{MyACRNema}                \PYG{o}{=} NNNNNNNN
\PYG{n+nv}{TCPPort}                  \PYG{o}{=} \PYG{l+m}{5678}

\PYG{c+c1}{\PYGZsh{} Host for postgres or mysql only, name, username and password for database}
\PYG{n+nv}{SQLHost}                  \PYG{o}{=} localhost
\PYG{n+nv}{SQLServer}                \PYG{o}{=} conquest
\PYG{n+nv}{Username}                 \PYG{o}{=} conquest
\PYG{n+nv}{Password}                 \PYG{o}{=} xxxxxxxxx
\PYG{n+nv}{PostGres}                 \PYG{o}{=} \PYG{l+m}{1}
\PYG{n+nv}{MySQL}                    \PYG{o}{=} \PYG{l+m}{0}
\PYG{n+nv}{SQLite}                   \PYG{o}{=} \PYG{l+m}{0}
\PYG{n+nv}{DoubleBackSlashToDB}      \PYG{o}{=} \PYG{l+m}{1}
\PYG{n+nv}{UseEscapeStringConstants} \PYG{o}{=} \PYG{l+m}{1}

\PYG{c+c1}{\PYGZsh{} Configure server}
\PYG{n+nv}{ImportExportDragAndDrop}  \PYG{o}{=} \PYG{l+m}{1}
\PYG{n+nv}{ZipTime}                  \PYG{o}{=} \PYG{l+m}{05}:
\PYG{n+nv}{UIDPrefix}                \PYG{o}{=} \PYG{l+m}{99999}.99999
\PYG{n+nv}{EnableComputedFields}     \PYG{o}{=} \PYG{l+m}{1}

\PYG{n+nv}{FileNameSyntax}           \PYG{o}{=} \PYG{l+m}{4}

\PYG{c+c1}{\PYGZsh{} Configuration of compression for incoming images and archival}
\PYG{n+nv}{DroppedFileCompression}   \PYG{o}{=} un
\PYG{n+nv}{IncomingCompression}      \PYG{o}{=} un
\PYG{n+nv}{ArchiveCompression}       \PYG{o}{=} as

\PYG{c+c1}{\PYGZsh{} For debug information}
\PYG{n+nv}{PACSName}                 \PYG{o}{=} NNNNNNNN
\PYG{n+nv}{OperatorConsole}          \PYG{o}{=} \PYG{l+m}{127}.0.0.1
\PYG{n+nv}{DebugLevel}               \PYG{o}{=} \PYG{l+m}{0}

\PYG{c+c1}{\PYGZsh{} Configuration of disk(s) to store images}
\PYG{n+nv}{MAGDeviceFullThreshold}   \PYG{o}{=} \PYG{l+m}{30}
\PYG{n+nv}{MAGDevices}               \PYG{o}{=} \PYG{l+m}{1}
\PYG{n+nv}{MAGDevice0}               \PYG{o}{=} /opt/conquest\PYGZhy{}14\PYGZhy{}19b/data/

\PYG{c+c1}{\PYGZsh{} Files to store logs}
\PYG{n+nv}{StatusLog} \PYG{o}{=} /opt/conquest\PYGZhy{}14\PYGZhy{}19b/logs/serverstatus.log
\PYG{n+nv}{TroubleLog} \PYG{o}{=} /opt/conquest\PYGZhy{}14\PYGZhy{}19b/logs/pacstrouble.log
\PYG{n+nv}{UserLog} \PYG{o}{=} /opt/conquest\PYGZhy{}14\PYGZhy{}19b/logs/pacsuser.log
\end{sphinxVerbatim}

\fvset{hllines={, ,}}%
\begin{sphinxVerbatim}[commandchars=\\\{\}]
vim /var/www/cgi\PYGZhy{}bin/dicom.ini
\end{sphinxVerbatim}

\fvset{hllines={, ,}}%
\begin{sphinxVerbatim}[commandchars=\\\{\}]
\PYG{o}{[}sscscp\PYG{o}{]}
\PYG{n+nv}{MicroPACS}                \PYG{o}{=} sscscp

\PYG{c+c1}{\PYGZsh{} database layout (copy dicom.sql to the web server script directory or point to the one in your dicom server directory)}

\PYG{n+nv}{kFactorFile}              \PYG{o}{=} /opt/conquest\PYGZhy{}14\PYGZhy{}19b/dicom.sql

\PYG{c+c1}{\PYGZsh{} gives access to the SQL server of the DICOM server}
\PYG{c+c1}{\PYGZsh{} use of independent database is also allowed (depends on scripts used)}

\PYG{n+nv}{SQLHost}                  \PYG{o}{=} localhost
\PYG{n+nv}{SQLServer}                \PYG{o}{=} conquest
\PYG{n+nv}{Username}                 \PYG{o}{=} conquest
\PYG{n+nv}{Password}                 \PYG{o}{=} xxxxxxxxx
\PYG{n+nv}{PostGres}                 \PYG{o}{=} \PYG{l+m}{1}
\PYG{n+nv}{MySQL}                    \PYG{o}{=} \PYG{l+m}{0}
\PYG{n+nv}{SQLite}                   \PYG{o}{=} \PYG{l+m}{0}
\PYG{n+nv}{DoubleBackSlashToDB}      \PYG{o}{=} \PYG{l+m}{1}
\PYG{n+nv}{UseEscapeStringConstants} \PYG{o}{=} \PYG{l+m}{1}

\PYG{c+c1}{\PYGZsh{} gives access to all DICOM servers known in acrnema.map}

\PYG{n+nv}{ACRNemaMap}               \PYG{o}{=} /opt/conquest\PYGZhy{}14\PYGZhy{}19b/acrnema.map
\PYG{n+nv}{Dictionary}               \PYG{o}{=} /opt/conquest\PYGZhy{}14\PYGZhy{}19b/dgate.dic
\PYG{n+nv}{SOPClassList}             \PYG{o}{=} /opt/conquest\PYGZhy{}14\PYGZhy{}19b/dgatesop.lst

\PYG{c+c1}{\PYGZsh{} default IP address and port of DICOM server (may be non\PYGZhy{}local, web pages empty if wrong)}

\PYG{n+nv}{WebServerFor}             \PYG{o}{=} \PYG{l+m}{127}.0.0.1
\PYG{n+nv}{TCPPort}                  \PYG{o}{=} \PYG{l+m}{5678}

\PYG{c+c1}{\PYGZsh{} AE title: only used if web client originates queries or moves}

\PYG{n+nv}{MyACRNema}                \PYG{o}{=} NNNNNNNN

\PYG{c+c1}{\PYGZsh{} path to script engine: ocx will not download images if wrong \PYGZhy{} shows as black square with controls}

\PYG{n+nv}{WebScriptAddress}         \PYG{o}{=} http://127.0.0.1/cgi\PYGZhy{}bin/dgate

\PYG{c+c1}{\PYGZsh{} if set to 1 (default), the web user cannot edit databases and (in future) other things}
\PYG{c+c1}{\PYGZsh{} webpush enables push of data to other servers}

\PYG{n+nv}{WebReadonly}              \PYG{o}{=} \PYG{l+m}{0}
\PYG{n+nv}{WebPush}                  \PYG{o}{=} \PYG{l+m}{1}
\end{sphinxVerbatim}

\fvset{hllines={, ,}}%
\begin{sphinxVerbatim}[commandchars=\\\{\}]
vim /var/www/cgi\PYGZhy{}bin/newweb/dicom.ini
\end{sphinxVerbatim}

\fvset{hllines={, ,}}%
\begin{sphinxVerbatim}[commandchars=\\\{\}]
\PYG{o}{[}sscscp\PYG{o}{]}
\PYG{n+nv}{MicroPACS}                \PYG{o}{=} sscscp
\PYG{n+nv}{ACRNemaMap}               \PYG{o}{=} acrnema.map
\PYG{n+nv}{Dictionary}               \PYG{o}{=} dgate.dic
\PYG{n+nv}{WebServerFor}             \PYG{o}{=} \PYG{l+m}{127}.0.0.1
\PYG{n+nv}{TCPPort}                  \PYG{o}{=} \PYG{l+m}{5678}
\end{sphinxVerbatim}


\subsection{Edit acrnema.map}
\label{\detokenize{trl2:edit-acrnema-map}}
For \sphinxcode{\sphinxupquote{acrnema.map}}, every files are identical in the content and should contain the correct AE name i.e. NNNNNNNN

\fvset{hllines={, ,}}%
\begin{sphinxVerbatim}[commandchars=\\\{\}]
NNNNNNNN                \PYG{l+m}{127}.0.0.1       \PYG{l+m}{5678}            un
\end{sphinxVerbatim}


\section{Running the DICOM server}
\label{\detokenize{trl2:running-the-dicom-server}}
Upon completion of the installation which includes compilation and settings, the database will need to be initialized for the DICOM server.

\fvset{hllines={, ,}}%
\begin{sphinxVerbatim}[commandchars=\\\{\}]
/opt/conquest\PYGZhy{}14\PYGZhy{}19b/dgate \PYGZhy{}v \PYGZhy{}r
\end{sphinxVerbatim}

Create init script for dgate service

\fvset{hllines={, ,}}%
\begin{sphinxVerbatim}[commandchars=\\\{\}]
vim /lib/systemd/system/dgate.service
\end{sphinxVerbatim}

\fvset{hllines={, ,}}%
\begin{sphinxVerbatim}[commandchars=\\\{\}]
\PYG{o}{[}Unit\PYG{o}{]}
\PYG{n+nv}{Description}\PYG{o}{=}Conquest DICOM Server dgate
\PYG{n+nv}{Documentation}\PYG{o}{=}https://ingenium.home.xs4all.nl/dicom.htm
\PYG{n+nv}{After}\PYG{o}{=}syslog.target
\PYG{n+nv}{After}\PYG{o}{=}network.target

\PYG{o}{[}Service\PYG{o}{]}
\PYG{n+nv}{Type}\PYG{o}{=}simple

\PYG{n+nv}{User}\PYG{o}{=}conquest
\PYG{n+nv}{Group}\PYG{o}{=}conquest

\PYG{n+nv}{ExecStart}\PYG{o}{=}/opt/conquest\PYGZhy{}14\PYGZhy{}19b/dgate \PYGZhy{}v \PYGZhy{}L/opt/conquest\PYGZhy{}14\PYGZhy{}19b/logs/serverstatus.log

\PYG{o}{[}Install\PYG{o}{]}
\PYG{n+nv}{WantedBy}\PYG{o}{=}multi\PYGZhy{}user.target
\end{sphinxVerbatim}

Enable and start the service

\fvset{hllines={, ,}}%
\begin{sphinxVerbatim}[commandchars=\\\{\}]
systemctl \PYG{n+nb}{enable} dgate.service
systemctl start dgate.service
systemctl status dgate.service
\end{sphinxVerbatim}

Now DICOM server is online and ready to receive connection. Set SELinux to Permissive before testing the web service.

\fvset{hllines={, ,}}%
\begin{sphinxVerbatim}[commandchars=\\\{\}]
vim /etc/selinux/config
\end{sphinxVerbatim}

\fvset{hllines={, ,}}%
\begin{sphinxVerbatim}[commandchars=\\\{\}]
\PYG{n+nv}{SELINUX}\PYG{o}{=}permissive
\PYG{n+nv}{SELINUXTYPE}\PYG{o}{=}targeted
\end{sphinxVerbatim}

Open the web service with any browser e.g. links to localhost

\fvset{hllines={, ,}}%
\begin{sphinxVerbatim}[commandchars=\\\{\}]
links http://127.0.0.1/cgi\PYGZhy{}bin/dgate?mode\PYG{o}{=}top
links http://127.0.0.1/cgi\PYGZhy{}bin/newweb/dgate
\end{sphinxVerbatim}



\renewcommand{\indexname}{Index}
\printindex
\end{document}